\section*{Liczby zespolone}

Najważniejsze własności:

\begin{itemize}
    \item $i^2 = -1$
    \item $e^{i\varphi} = cos{\varphi} + i\sin{\varphi}$
    \item $|e^{i\varphi}| = 1$
    \item $|z| = \sqrt{(\operatorname{re}{z})^2 +
    (\operatorname{im}{z})^2}$
    \item $z = |z|\cdot e^{i\varphi}$
    \item Liczba ma $n$ pierwiastków zespolonych $n$-tego stopnia.
\end{itemize}
Wzór na $k$-ty pierwiastek zespolony $n$-tego stopnia z liczby $z$:
\begin{equation*}
    \sqrt[n]{z} = \sqrt[n]{|z|}
    \left(
    \sin{\left(\frac{2k\pi}{n}\right)}
    + i\sin{\left(\frac{2k\pi}{n}\right)}
    \right)
\end{equation*}
gdzie
$k = \{1, 2, 3, \ \hdots\ n - 1\}$