\section*{Rozkłady}

\subsection*{Rozkład Gaussa}

Funkcja gęstości prawdopodobieństwa dana jest wzorem
\begin{equation*}
    f(x) = \frac{1 }{\sigma \sqrt{2\pi}} \cdot 
    \exp{\left(-\frac{(x-\mu)^2}{2\sigma^2}\right)}
\end{equation*}
gdzie
\begin{itemize}
    \item $\sigma$ - odchylenie standardowe
    \item $\mu$ - średnia
    \item $\exp{a} = e^a$
\end{itemize}

\subsubsection*{Dodawanie rozkładów}

Dana jest zmienna losowa $Y = X_1(\mu_1, \sigma_1) + X_2(\mu_1, \sigma_1) + \hdots + X_n(\mu_n, \sigma_n)$, gdzie rozkłady zmiennych $X_{1-3}$, są rozkładami normalnymi.\\

$\mu_y = \mu_1 + \mu_2 + \hdots + \mu_n$

$\sigma_y = \sqrt{\sigma_1^2 + \sigma_2^2 + \hdots \sigma_N^2}$
