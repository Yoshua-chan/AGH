\section*{Wielomiany Czebyszewa}
Wzor rekurencyjny:

\begin{itemize}
    \item $T_0(x) = 1$
    \item $T_1(x) = x$
    \item $T_n(x) = 2x\cdot T_{n-1}(x)-T_{n-2}(x)$, dla n $\ge$ 2 
\end{itemize}

Wielomiany Czebyszewa mają $n$-tego stopnia mają na przedziale $n$
równo rozmieszczonych miejsc zerowych określonych wzorem:
\begin{equation*}
    x_k = \cos{\left(\frac{\pi\left(k-0.5\right)}{n}\right)}
\end{equation*}

Oraz $n + 1$ ekstremów w punktach określonych wzorem
\begin{equation*}
    x_m = \cos{\left(\frac{\pi\cdot m}{n}\right)}
\end{equation*}

dla

$k = 1, 2\ \hdots\ n$

$m = 0, 1\ \hdots\ n$