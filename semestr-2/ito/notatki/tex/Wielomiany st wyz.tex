\section*{Rozwiązywanie wielomianów stopni wyższych}
\subsection*{Wzory Cardano}
Wzory Cardano pozwalają obliczyć pierwiastki wielomianu trzeciego stopnia,
jeśli wielomian jest unormowany, czyli współczynnik przy największej
potędze jest równy $1$.\\
Dla wielomianu w postaci $x^3 + a_2x^2 + a_1x + a_0 = 0$ wzory
Cardano mają następującą postać:
\begin{equation*}
    y = x + \frac{a_2}{3} \Rightarrow y^3 + (3p) \cdot y + 2q = 0
\end{equation*}

\begin{equation*}
    p = \frac{3a_1 - a_2^2}{9}\ \ \ 
    q = \frac{a_2^3}{27} - \frac{a_1 a_2}{6} + \frac{a_0}{2}
\end{equation*}

\begin{equation*}
    D = q^2 + p^3
\end{equation*}

Jeśli $D > 0$, to $y_1 \in \mathbb{R} \land y_2, y_3 \in \mathbb{C} \setminus \mathbb{R}$.\\
Stosujemy następującą metodę:

\begin{equation*}
    u = \sqrt[3]{- q + \sqrt{D}}\ \ \ 
    v = \sqrt[3]{- q - \sqrt{D}}
\end{equation*}

\begin{equation*}
    y_1 = u + v,\ \ y_{2,3} = -\frac{1}{2}\left(u+v\right)
    \pm j \cdot \frac{\sqrt{3}}{2}\left(u-v\right)
\end{equation*}

Jeśli $D < 0$, czyli $p < 0$, to wszystkie miejsca zerowe $y_k$ są rzeczywiste i różne.\\
Liczymy je w następujący sposób:

\begin{equation*}
    y_k = 2 \sqrt{-p} \cdot \cos{\left(\frac{\varphi + 2\pi\left(k-1\right)}
    {3}\right)}
\end{equation*}

gdzie
$$ 
\varphi \in \left<0; \pi\right>,\ \cos{\varphi} = \frac{-q}{\sqrt{-p^3}}
$$\\

Ostatecznie podstawiamy z powrotem pod $x$:

\begin{equation*}
    x = y - \frac{a_2}{3}
\end{equation*}


\subsection*{Wzory Ferrari}

Dany jest wielomian o wzorze $x^4 + a_3x^3 + a_2x^2 + a_1x = a_0$.

\begin{enumerate}
    \item Podstawiamy pod wielomian 3 stopnia według wzoru
    \begin{equation*}
        k^3 + \left(-\frac{a_2}{2}\right)\cdot k^2 +
        \left(\frac{a_3 a_1 - 4a_0}{4}\right)\cdot k +
        \frac{4a_2a_0 - a_3^2 a_0 - a_1^2}{8} = 0
    \end{equation*}
    \item Liczymy dowolny pierwiastek powyższego wielomianu i oznaczamy jako $k$. Następnie podstawiamy według wzorów:
    \begin{equation*}
        p = \sqrt{2k + \frac{a_3^2}{4} - a_2},\ \ q = \frac{k\cdot a_3 - a_1}{2p}
    \end{equation*}
    \item Podstawiamy do równania:
    \begin{equation*}
        x^2 + \left(\frac{a_3}{2} \pm p\right)\cdot x + k \pm q = 0
    \end{equation*}
    \item Następnie rozwiązujemy dwa równania i mamy pierwiastki
\end{enumerate}