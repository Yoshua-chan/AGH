\section*{Rozwiązywanie wielomianów stopni wyższych}
\subsection*{Wzory Cardano}
Wzory Cardano pozwalają obliczyć pierwiastki wielomianu trzeciego stopnia,
jeśli wielomian jest unormowany, czyli współczynnik przy największej
potędze jest równy $1$.\\
Dla wielomianu w postaci $x^3 + a_2x^2 + a_1x + a_0 = 0$ wzory
Cardano mają następującą postać:
\begin{equation*}
    y = x + \frac{a_2}{3}
\end{equation*}

\begin{equation*}
    p = \frac{3a_1 - a_2^2}{9}\ \ \ 
    q = \frac{a_2^3}{27} - \frac{a_1 a_2}{6} + \frac{a_0}{2}
\end{equation*}

\begin{equation*}
    D = q^2 + p^3
\end{equation*}

\subsection*{Wzory Ferrari}